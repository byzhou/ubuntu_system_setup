\documentclass[]{article}
\usepackage{graphicx}

\parindent=0pt
\usepackage[margin=0.5in]{geometry}

\begin{document}
\pagestyle{empty}
{\large\textbf{Research Notes}}
\begin{itemize}
    \item[*] Created on Nov 3, 2014
    \item[*] Modified on \today
    \item[*] Author info: Boyou Zhou\\
             8 St Mary's St, PHO 340, Boston, MA 02215\\
             Email: bobzhou@bu.edu, Phone: 617-678-8480
\end{itemize}


\rule[-0.1cm]{7.5in}{0.01cm}\\
\\
\noindent \textbf{Nov 3,2014}
\textit{A completely on-chip switched-capacitor DC-DC Converter using digital
capacitance modulation for LDO replacement in 45nm CMOS}
\indent		\begin{itemize}
            \item \textit{motivation} Traditional LDO can not be applied to low
            voltages such as voltages below 1.2 voltages. The capacitors can be
            applied for such low voltages. The caps occupies a large area on
            chip so it is not poplular.
            \item \textit{DCM} Digital Capacitance Modulation
            \item \textit{main idea} The author has allowed coarse tuning and
            fine tunning to achieve LDO. THe frequency is determined by the
            load. The author claimed the efficiency is more than $i60\%$.
        \end{itemize}


\noindent \textbf{Nov 4,2014}
\textit{A Sub-1 V Transient-Enhanced Output-Capacitorless LDO Regulator With
Push-Pull Composite Power Transistor}
\indent		\begin{itemize}
            \item \textit{motivation} The proposed idea uses the nondominant
            parastic poles that can push the working frequency higher.
            \item \textit{Get rid of oscillation} Get rid of high-impedance
            points in the loop.
            \item \textit{Benefits} The only difference is that the author
            implimented the entire design with a ClassA power amplifier instead
            of a regular current source, which is the push part of the circuit.
            \item {Critics} The LDO regulation starts too slow.
       		\end{itemize}


\noindent \textbf{Nov 4,2014}
\textit{High PSR Low Drop-Out Regulator With Feed-Forward Ripple Cancellation
Technique}
\indent		\begin{itemize}
            \item {main motivation}Conventional LDO has poor performance over
            $300kHz$ with sub-250nm technology. In order to improve the PSR,
            there are several basic techiniques. 
                \begin{itemize}
                    \item Using simple RC filtering at the output of LDO.
                    \item Cascading.
                    \item Cascading another transistor with the pMOS along with
                    RC filtering.
                \end{itemize}
            \item {main idea} The author combine the idea of feedforward with
            low pass filtering into the common LDO. One path is the feedforward
            path that has cancellation path feeding into the op-amp. Another is
            the feedback path withthe error amplifying into the op-amp.
            \end{itemize}

\end{document}

