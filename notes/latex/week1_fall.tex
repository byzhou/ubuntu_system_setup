\documentclass[]{article}
\usepackage{graphicx}

\parindent=0pt
\usepackage[margin=0.5in]{geometry}

\begin{document}
\pagestyle{empty}
{\large\textbf{Research Notes}}
\begin{itemize}
    \item[*] Created on Sept 3, 2014
    \item[*] Modified on \today
    \item[*] Author info: Boyou Zhou\\
             8 St Mary's St, PHO 340, Boston, MA 02215\\
             Email: bobzhou@bu.edu, Phone: 617-678-8480
\end{itemize}


\rule[-0.1cm]{7.5in}{0.01cm}\\
\\
\noindent \textbf{Sept 3,2014}
\textit{A 10.4pJ/b (32,8) LDPC Decoder with Time-Domain Analog and Digital Mixed-Signal Processing}
\indent		\begin{itemize}
			\item \textit{Main motivation}  Most of the analog computation circuits have a much higher efficiency in computational operation.
            The paper argued that the time-domain analog and digital mixed signal will be more effective in the computation. 
            Thus the paper has used a LDPC decoder as the example.

			\item \textit{LDPC decoder} LDPC is a very efficient coding method that has been applied in Wifi standard.
            The basic idea is to multiply the info that needs to be coded with a polarity check matrix to form the coded info.
            The circuit is very easy to design, as it just consists of an adder and multiplication and a feedback loop.
            The decoding part is using possibility decoding, so it will involve a little programming.
            I think LDPC decoder can be a small project with the matlab decoding.

            \item \textit{Time-to-digital} Use a counter to count how many clock cycles have passed to measure the time that has passed.
            And this value is converted to a digital representation. 
            TDC requires highly accurate clock in order to achieve correct time to digital conversion.
            The fine measurements have not been read in this review. \textbf{Sept 3, 2004}

            \item \textit{Basic idea} The info that needs to be coded will be go into a DTC and converted to time info. 
            It will than going through a special designed LDPC decoder in which basic operations have been done in the analog domain.
            And then convert the info to the digital domain. 
            This wil reduce energy consumption, because all info have been done in the analog domain.

            \item \textit{Critics} The testbench is only for the LDPC decoder, for some other circuit it might not need such amount of clock signal transition.
            From a personal point of view, it is strongly recommanded for other circuit for testing. 
            It is a perfect idea for applying analog computional circuit to digital domain.
            But overhead might be a problem in other circuits, especially for the DTC and TDC.

        \end{itemize}


\noindent \textbf{Sept 4,2014}
\textit{Self-Super-Cutoff Power Gating with State Retention on a 0.3V 0.29fJ/Cycle/Gate 32b RISC Core in 0.13um}
\indent		\begin{itemize}
            \item \textit{Main Motivation} The paper proposed an idea to clamp the leakage power under super low supply voltage.

            \item \textit{Basic idea} 
            Compared to the traditional ways to cut-off the supply voltage, this paper illustrates the idea of implement with two different supply voltage.
            The point to try to cut off supply voltage in order to save energy while the processor is at standing by state.
            The system needs a strong clock signal, thus the author proposed that the keeper for the clock buffer should be supplied with vdd.

            \item \textit{Galios Field} 
            \textbf{LDPC code book is based on GF}
            In algebra, the Galios Field is a finite set that consists of limited elements.
            The number of the elements is called order.
            The set is defined with limited number of operations, such as addition, subtraction, multiplication, and division.
            Finite fields only exits when the order is a prime number $p^{k}$.
            For eack given k, the field is isomorphic.
            The characteristic of the field is p, meaning any p copies sum is zero.

            \item \textit{Critics} 
            The question is that when the system should have a stand-by state so that it can save more energy.

        \end{itemize}

\noindent \textbf{Sept 5,2014}
\textit{Low Power 1G Razor FIR Accelerator with Time borrowing tracking pipeline and approximate error correction}
\indent		\begin{itemize}

            \item \textit{Main Motivation} It's Razor, you know it.
            This time these guys are thinking about how to reduce the overhead compared to the ANT circuits.
            They want to demo it on a DSP chip.
            They claim that they can correct the error on a algorithm-level.

            \item \textit{ANT} two papers about Antinoise noise tolerance circuits \textbf{not looking into it, but sounds interesting, Sept 5,2014}\\
            one, A Voltage Overscaled low-power digital filter IC\\
            two, low power and error resilient PN code Acquisition filter via statistical error compensation\\
            I mean, we should have published this thing faster than these UIUC guys.

            \item \textit{main idea} Still, it is using the RZFF to detect errors and correct them. 
            This time it use a FIFO to do the linear prediction in order to predict the correction in order to achieve a higher efficiency in improving accuracy.
            
            \item \textit{Critics}   I think the error detection can be implement with cirtain linear prediction.
            Because this is the a FIR, of course, the correlation between the stream of bits should be much stronger.
            Thus, I think, prediction can be applied to both detection and correction.
            However, applying to detection might have to pay a higher price in cost. 

        \end{itemize}
\end{document}

